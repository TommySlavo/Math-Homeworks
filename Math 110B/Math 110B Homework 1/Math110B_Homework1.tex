\documentclass{article}

\usepackage[utf8]{inputenc}
\usepackage{mathrsfs}
\usepackage{tikz}
\usepackage{amssymb}
\usepackage{amsthm}
\usepackage{graphicx} % Required for inserting images
\usepackage{amsmath}
\usepackage{MnSymbol}
\usepackage{geometry}
\usepackage{physics}
\usepackage{enumerate}
\usepackage{enumitem}
\newcommand\numeq[1]%
  {\stackrel{\scriptscriptstyle(\mkern-1.5mu#1\mkern-1.5mu)}{=}}
\geometry{a4paper, margin=1in} % set margins

\title{Math 132H Homework 1}
\author{Thomas Slavonia; UID: 205511702}
\date{\today}
\begin{document}
\maketitle

\section*{1.}
We begin by noting that $|S_3| = 3!$, so apart from $\phi$ and $\psi$, we need to find four more elements of $S_3$ by taking the compositions of $\phi$ and $\psi$.
\[
  \phi \circ \phi = \begin{bmatrix}
    1 & 2 & 3 \\
    1 & 2 & 3
  \end{bmatrix}
  \]
\[
  \psi \circ \psi = \begin{bmatrix}
    1 & 2 & 3 \\
    3 & 1 & 2
  \end{bmatrix}
\]
\[
\phi \circ \psi = \begin{bmatrix}
  1 & 2 & 3 \\
  1 & 3 & 2
\end{bmatrix}  
\]
\[
\psi \circ \phi = \begin{bmatrix}
  1 & 2 & 3 \\
  3 & 2 & 1 
\end{bmatrix}.
\]

\section*{2.}
\begin{proof}
  Let $G$ and $H$ be groups with operations $*$ and $\cdot $ respectively and $G$ and/or $H$ nonabelian. Define the operation of the group $G \times H$ as \[(g, h) \circ (g', h') = (g * g', h \cdot h').\] 
  We showed in lecture that $G \times H$ is indeed a group. Also, the group $G \times H$ is also nonabelian, as if either $g*g' \neq g'*g$ or $h \cdot h' \neq h' \cdot h$, then 
  \[
    (g, h) \circ (g', h') = (g * g', h \cdot h') \neq (g' *g , h' \cdot h) = (g', h') \circ (g, h).
    \]

    We must show that $|G \times H| = |G||H|$. To show this, we can note that for the first component of any element of $G \times H$, we have $|G|$ choices, and similarly, for the second component, we have $|H|$. Thus, we have that $|G \times H| = |G||H|$. We know from the textbook that $D_4$ is a nonabelian group; therefore, $D_4 \times D_4$ is a nonabelian group of order $16$. Similarly, using that $G \times H$ is nonabelian if either $G$ or $H$ is nonabelian, we can define $D_4 \times \mathbb{Z}_6$ is a nonabelian group of order $48$. Lastly, note that $D_3 = \{r_0, r_1, r_2, s, t, u\}$ is nonabelian as $r_1 s \neq s r_1$ (further visualization in the book), so we have $D_3 \times \mathbb{Z}_2$ and $D_4 \times \mathbb{Z}_5$ as nonabelian groups of order $12$ and $30$ respectively. 

  
\end{proof}

\section*{3.}
\subsection*{a.}
\begin{proof}
 Let $a \in G$ and suppose $|a| = 12$. Let $n = 12 = |a|$. We will apply the theorem that states that if $n = td$ for $d \geq 1$, then $a^t$ has order $d$ repeatedly throughout this problem. The theorem applies to the following:
 \begin{align*}
  12 &= 1 \cdot 12 \Rightarrow |a| = 12 \\
  12 &= 2 \cdot 6 \Rightarrow |a^2| = 6 \\
  12 &= 3 \cdot 4 \Rightarrow |a^3| = 4 \\
  12 &= 4 \cdot 3 \Rightarrow |a^4| = 3 \\
  12 &= 6 \cdot 2 \Rightarrow |a^6| = 2.
 \end{align*}
We now know that $|a^3| = 6$ and that $(a^3)^3 = a^9$. Note that $a^k = e$ if and only if $n | k$, thus, $a^{12} = e$, $a^{24} = e$, $a^36 = e$, $a^{48} = e$, $a^{60} = e$, $a^{84} = e$, and $a^{132} = e$. Thus, the least common multiples of $n$ and $5, 7, 8, 9, 10, 11$ and $12$ are $60, 84, 24, 36, 60, 132$ respectively. 
Now, we know that $a^{60} = \left(a^5\right)^{12}$, $a^{84} = \left(a^7 \right)^{12}$, $a^{24} = \left(a^8 \right)^3$, $a^{36} = \left(a^9 \right)^4$, $a^{60} = \left(a^{10} \right)^6$, and $a^{132} = \left(a^{11} \right)^{12}$. This all implies that $|a^5| = 12$, $|a^7| = 12$, $|a^8| = 3$, $|a^9| = 4$, $|a^{10}| = 6$, and $|a^{11}| = 12$. 
\end{proof}

\subsection*{b.}
\begin{proof}
  My conjecture on the order of $a^k$ when $|a| = n$ is that is the order of $|a^k| = \frac{n}{gcd(k, n)}$.  
\end{proof}

\section*{4.}

\begin{proof}
Let $G = \{a_1, a_2, \ldots , a_n \}$ be a finite abelian group of order $n$. Let $x = a_1a_2 \cdots a_n$. 
Thus, $x^2 = a_1a_2 \cdots a_n a_1 a_2 \cdots a_n$. Every element of a group must have an inverse; therefore, $a_i^{-1}$ is represented somewhere in the product $a_1 a_2 \cdots a_n$. Because the group is commutative, we can commute the elements of the sum (which are all elements of the group) to get that $a_1a_2 \cdots a_n = a_1^{-1}a_2^{-1} \cdots a_n^{-1}$. 
Now, $x^2 = a_1a_2 \cdots a_n a_1^{-1}a_2^{-1} \cdots a_n^{-1}$ which we can once again use the commutative of the group to rewrite this as $x^2 = a_1a_1^{-1}a_2a_2^{-1} \cdots a_na_n^{-1} = e$. 
\end{proof}

\section*{5.}
\begin{proof}
Suppose $G$ is a group where every nonidentity element has order $2$. Take $a, b \in G$. We want to show that $ab = ba$. Using our assumption, we find 
\[
e = a^2 = aa = aea = ab^2a = abba     
\]
which implies 
\[
  a^{-1}b^{-1} = ab
\]
and thus 
\[
  a^{-1} = abb = ab^2 = ae = a.
\]
Note that every element of a group has a unique inverse and is closed under multiplication; hence, for $ab \in G$, $(ab)^{-1} = b^{-1}a^{-1}$ as $abb^{-1}a^{-1} = e$. We also now know that every element of the group is equal to its inverse. Therefore, we can conclude that $ab = b^{-1}a^{-1}$, $a = a^{-1}$, and $b = b^{-1}$ and get the desired result: 
\[
  ab = b^{-1}a^{-1} = ba. 
\]
\end{proof}

\section*{6.}
\begin{proof}
Suppose, for a group $G$, that $(ab)^i = a^ib^i$ for three consecutive integers $i$ and every $a, b \in G$. Take $a, b \in G$. Also, without loss of generality, suppose the consecutive integers are $i$, $i + 1$, and $i + 2$. We use these properties to see 
\[
 a^{i + 1}b^{i + 1} = (ab)^{i + 1} = ab(ab)^i = aba^ib^i 
\]
which implies that 
\[
 a^{i + 1}b = aba^i 
\]
and by multiplying on the right by $a^{-1}$ we get
\[
 a^ib = ba^i.
\]
Similarly,
\[
  a^{i + 2}b^{i + 2} = (ab)^{i + 2} = ab(ab)^{i + 1} = aba^{i + 1}b^{i + 1}.
\]
which implies 
\[
 a^{i + 1}b = ba^{i + 1}.  
\]
 Thus, we can find
\[
 ab = aba^ia^{-i} =  aa^iba^{-i} = a^{i + 1}ba^{-i} = ba^{i + 1}a^{-i} = ba. 
\]
\end{proof}

\end{document}
