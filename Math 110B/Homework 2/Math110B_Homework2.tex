\documentclass{article}
\usepackage[utf8]{inputenc}
\usepackage{mathrsfs}
\usepackage{tikz}
\usepackage{amssymb}
\usepackage{amsthm}
\usepackage{graphicx} % Required for inserting images
\usepackage{amsmath}
\usepackage{MnSymbol}
\usepackage{geometry}
\usepackage{physics}
\usepackage{enumerate}
\allowdisplaybreaks
\newcommand\numeq[1]%
  {\stackrel{\scriptscriptstyle(\mkern-1.5mu#1\mkern-1.5mu)}{=}}
\newcommand\numleq[1]
  {\stackrel{\scriptscriptstyle(\mkern-1.5mu#1\mkern-1.5mu)}{\leq}}
\newcommand\numgeq[1]
  {\stackrel{\scriptscriptstyle(\mkern-1.5mu#1\mkern-1.5mu)}{\geq}}

\newtheorem{definition}{Definition}[section]
\newtheorem{theorem}{Theorem}[section]
\newtheorem{remark}{Remark}[section]
\newtheorem{example}{Example}[section]



\title{Math 110B Homework 2}
\author{Thomas Slavonia}
\date{\today}

\begin{document}
\maketitle
\section*{1.}
\subsection*{a.}
\begin{proof}
  Let $H$ and $K$ be subgroups of $G$. Since $e \in H$ and $e \in K$, then $e \in H \cap K$. Take $a \in H \cap K$. Then, $a \in H$ and $a \in K$. Since $H$ and $K$ are subgroups we have $a^{-1} \in H$ and $a^{-1} \in K$ and thus $a^{-1} \in H \cap K$. Take $a, b \in H \cap K$. Then, $a, b \in H$ and $a, b \in K$. Note that $H$ and $K$ are subgroups, so $ab \in H$ and $ab \in K$, and therefore $ab \in H \cap K$. Thus, by Theorem 7.11 in the book these are the only axioms we needed to satisfy for $H \cap K$ to be a subgroup, thus $H \cap K$ is a subgroup. 
\end{proof}
\subsection*{b.}
\begin{proof}
  Let $\{H_i\}$ be a collection of subgroups of $G$. Take $a \in \cap H_i$. Then, we have that $a \in H_i$ for any $i$. Since $H_i$ is a subgroup we know that $a^{-1} \in H_i$ for every $i$ and thus we can conclude that $a^{-1} \in \cap H_i$. Take $a, b \in \cap H_i$. Hence, we have that $a, b \in H_i$ for every $i$. Since $H_i$ is a subgroup, we know that $ab \in H_i$ for every $i$ and thus we can conclude that $ab \in \cap H_i$. Thus, by Theorem 7.11 in the book these are the only axioms we needed to satisfy for $\cap H_i$ to be a subgroup, thus $\cap H_i$ is a subgroup.
\end{proof}
\section*{2.}
\begin{proof}
  Let $H$ be a subgroup of $G$. Note, that the normalizer of $H$ is defined as $N(H) = \{x \in G : xHx^{-1} = H \}$. Now, take $a \in N(H)$. Therefore
  \[
  aHa^{-1} = H  
  \] 
  but, if we multiply by $a^{-1}$ on the left and $a$ on the right that 
  \[
  a^{-1}aHa^{-1}a = H = a^{-1}Ha  
  \]
  and thus $a^{-1} \in N(H)$. Take $a, b \in N(H)$. Then, 
  \[
  aHa^{-1} = H, \text{ and } bHb^{-1} = H.  
  \]
  Then using the proven fact (Corollary 7.6 in the book) that $(ab)^{-1} = b^{-1}a^{-1}$ we get the result
  \[
  abH(ab)^{-1} = abHb^{-1}a^{-1} = aHa^{-1} = H.   
  \]
  Thus, $ab \in N(H)$ and by Theorem 7.11 showing that a subset is closed and under operation and inverses is necessary to prove that a subset of a group is a subgroup. Let $a \in H$. Then, 
  \[
  aHa^{-1} = aH = H  
  \]
  and so $a \in N(H)$ and therefore $H \subset N(H)$. 
\end{proof}
\section*{3.}
\begin{proof}
  Let $G$ be an abelian group of order $mn$ where $(m, n) = 1$ with an element $a$ of order $m$ and an element $b$ of order $n$. Look at the subgroup $\langle ab \rangle$. Then, $(ab)^k = a^kb^k$ as the group is abelian. Sicne we can choose any $k \in \mathbb{Z}$ we could choose $k = 0$, so $\exists k \in \mathbb{Z}$ such that $(ab)^k = a^kb^k = e$ and thus $a^k = b^{-k}$. Take each side to the power of $n$ to get 
  \[
  \left(a^k\right)^n = \left(b^{-k} \right)^n  
  \] 
  \[
  a^{kn} = b^{-kn}.  
  \]
  $b$ is order $n$, so $b^{-kn} = e$ and so $a^{kn} = e$. $a$ is of order $m$, and therefore $m|kn$. Since $(m, n) = 1$, then $\exists c,d \in \mathbb{Z}$ such that $mc + nd = 1$. Let $kn = mq$ for $q \in \mathbb{Z}$ as $m|kn$. Then, multiplying by $k$ we get $mkc + knd = k$ and so $mkc + mqd = m(kc + qd) = k$. Therefore, as $(m, n) = 1$ we have that $m|k$ since $m|kn$. Thus, $a^k = e$. By our earlier equation we have 
  \[
  (a^k)^m = (b^{-k})^m  
  \]
  \[
  a^{km} = b^{-km}.
  \]
  $a$ is of order $m$, so $b^{-km} = e$. $b$ is of order $n$, and thus $n|km$ but since $(m, n) = 1$, using the previous argument we can say that $n|k$ also. Since $(m, n) = 1$ we have previously shown in algebra that if $n|k$ and $m|k$ and $n, m$ are coprime, then $nm|k$. Thus, $(ab)^{nm} = e$, so $ab$ has order $nm$. But, the order of $G$ is also $nm$ and thus it must be that $G = \langle ab \rangle$. 
\end{proof}
\section*{4.}
\subsection*{a.}
\begin{proof}
  Let $f:G \to H$ be a group homomorphism and $a \in G$ has finite order $k$. Then, 
  \[
  f(a)^k = \underbrace{f(a)\cdot f(a) \cdots f(a)}_{k-\text{times}} \numeq{a} f(\underbrace{a \cdot a \cdots a}_{k-\text{times}}) = f(a^k) \numeq{b} f(e_G) \numeq{c} e_H   
  \] 
  with steps $(a) -(c)$ justified:
  \begin{enumerate}[\indent (a)]
   \item $f$ is a group homomophism
   \item $a$ has order $k$ in $G$
   \item by theorem in book, identity element maps to identity element.  
  \end{enumerate} 
\end{proof}
\subsection*{b.}
\begin{proof}
  We know that $f(a)^k = e_H$. Therefore, $k$ is either the order of $f(a)$, or the order of $f(a)$ divides $k$, and either way we get the result that $|f(a)| \leq |a|$. 
\end{proof}
\section*{5.}
\begin{proof}
 Let $f:G \to H$ be a group homomorphism and $K_f = \{a \in G: f(a) = e_H \}$ be the kernel of the homomorphism. For $e_G \in G$ we know that $f(e_G) = e_H$ by a theorem in the book. Take $a \in K_f$, then the same theorem also gives us that $f(a^{-1}) = f(a)^{-1}$. Hence
 \[
 f(a^{-1}) = f(a)^{-1} = e_H^{-1} = e_H 
 \]
 which gives us that $f(a^{-1}) \in K_f$. Take $a, b \in K_f$. Then, 
 \[
 f(ab) = f(a)f(b) = e_He_H = e_H. 
 \]
 Thus, we have that $ab \in K_f$. These properties give us that $K_f$ is subgroup. 
\end{proof}
\section*{6.}
\begin{proof}
  Note that $\mathbb{Z}/n\mathbb{Z}$ is a cyclic group as $[1]$ will always be a generator as $(1, n) = 1$ always. Let $[a] \in \mathbb{Z}/n\mathbb{Z}$ be a generator for the group. Let $f \in \text{Aut }\mathbb{Z}/n\mathbb{Z}$. Then, $f([a]) = [ba]$ for $[b] \in \mathbb{Z}/n\mathbb{Z}$. If generators don't map to generators under an isomorphism, then the group structure is no longer maintained, thus $[ba]$ must be a generator of $\mathbb{Z}/n\mathbb{Z}$. So, $(ba, n ) = 1$ and $(a, n) = 1$ and so 
  \[
   bac + nd = 1 \text{ and } ax + ny = 1 
  \]
  for some $c, d, x, y \in \mathbb{Z}$. Therefore, 
  \[
   (bac + nd)(ax + ny) = bacax + bacny + ndax + ndny = b(acax + acny) + n(dax + dny) = 1 
  \]
  so $(b, n) = 1$. So, $[a], [b] \in U_n$. Note that for $\mathbb{Z}/n\mathbb{Z}$ we have that $1$ is always a generator, as $(1, n) = 1$ is always true. Look at the map $f:\text{Aut }\mathbb{Z}/n\mathbb{Z} \to U_n$ where for $\rho \in \text{Aut }\mathbb{Z}/n\mathbb{Z}$, $f(\rho) = [b]$ where we consider $b$ to be the integer that $1$ is shifted by. The map is well defined as if $\rho = \phi$, then $f(\rho) = [b] = f(\phi)$. Suppose $f(\rho) = f(\phi)$, then they both map to $[b]$, implying the $\rho = \phi$ as they both map $[1]$ to the same value and that will determine where all other values are mapped to since $[1]$ is a generator and thus $[b]$ will be a generator. Thus, the function is injective. Let $[b] \in U_n$. Then $[b]$ is a generator, so there exists an automorphism that maps $[1]$ by that generator as generators must map to generators in automorphisms. Thus, the function is surjective. Hence $f$ is an isomorphism and $\text{Aut }\mathbb{Z}/n\mathbb{Z} \cong U_n$. 
\end{proof}

\end{document}