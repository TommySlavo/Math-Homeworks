\documentclass{article}

\usepackage[utf8]{inputenc}
\usepackage{mathrsfs}
\usepackage{tikz}
\usepackage{amssymb}
\usepackage{amsthm}
\usepackage{graphicx} % Required for inserting images
\usepackage{amsmath}
\usepackage{MnSymbol}
\usepackage{geometry}
\usepackage{physics}
\usepackage{enumerate}
\newcommand\numeq[1]%
  {\stackrel{\scriptscriptstyle(\mkern-1.5mu#1\mkern-1.5mu)}{=}}
\newcommand\numleq[1]
  {\stackrel{\scriptscriptstyle(\mkern-1.5mu#1\mkern-1.5mu)}{\leq}}
\newcommand\numgeq[1]
  {\stackrel{\scriptscriptstyle(\mkern-1.5mu#1\mkern-1.5mu)}{\geq}}

\newtheorem{definition}{Definition}[section]
\newtheorem{theorem}{Theorem}[section]
\newtheorem{remark}{Remark}[section]
\newtheorem{example}{Example}[section]



\title{Math 110B Homework 4}
\author{Tom Slavonia}
\date{\today}

\begin{document}

\maketitle
\section*{1.}
\begin{proof}
    Let $Nx \in G/N$ for $x \in G$. Then, since $N$ is a normal subgroup and $x^2 \in N$ we have
    \[
    NxNx = NxxN = Nx^2N = NN = N.     
    \]
    Thus, implying all elements of $G/N$ have order $2$. 
\end{proof}

\section*{2.}
\begin{proof}
We look at $T \leq G$ the subgroup of elements of $G$ of finite order. The elements of $G/T$ are $Tx$ where $x \in G$. Note, since the group is abelian, we have that $xT = Tx$. If $x$ has finite order, then $x \in T$, so clearly $Tx = T$ has finite order but is also the idenity. Now, suppose $x$ has infinite order. Suppose for sake of contradiction $Tx$ has finite order $n$. Then, 
\[
 \underbrace{Tx \cdot Tx \cdots Tx}_{n-\text{times}} = T   
\]
but since the group is abelian so is $G/T$ by theorem 8.13 in the book
\[
    \underbrace{Tx \cdot Tx \cdots Tx}_{n-\text{times}} = = Tx^nT \cdots T = T 
\]
implying $x$ has finite order, but this is a contradiction as we assumed $x$ to have infinite order. Therefore, every nontrivial element of $G/T$ has infinite order. 
\end{proof}
\section*{3.}
\begin{proof}
    Suppose $N$ is a normal subgroup of $G$ such that $N$ and $G/N$ are finitely generated. Since $N$ is normal $G/N$ is a subgroup and is finitely generated, so $G/N = \langle Ng_1, Ng_2, \ldots, Ng_n \rangle$ for $n$ finite. The, we have 
    \[
    G = \cap_{i = 1}^n Ng_i    
    \]
    but, $N$ is finitely generated, and so it must be that $G$ is finitely generated. 
\end{proof}

\section*{4.}
\subsection*{a.}
\begin{proof}
Suppose $a \in K$. Therefore, \[\left(a^{-1}\right)^2 = a^{-2} = \left(a^{2}\right)^{-1} = e\]
so $K $ is closed under inverses. Take $a, b \in K$. Then, we have that, since the group is abelian
\[
   (ab)^2 = abab = aabb = a^2b^2 = ee = e.  
\]
Thus, $K$ is closed under multiplication and inverses which is sufficient for it to be a subgroup. 
\end{proof}
\subsection*{b.}
\begin{proof}
    Suppose $a \in H$. We then have that $a = x_1^2$ for some $x_1 \in G$. Therefore, 
    \[
    a^{-1} = \left(x_1^2\right)^{-1} = x_1^{-2} = (x_1^{-1})^2    
    \]
    and so $a^{-1} = \left(x^{-1} \right)^2$ implying $a^{-1} \in H$. Now, look at $a, b \in H$. Hence $a = x_1^2$ and $b = x_2^2$ for some $x_1, x_2 \in G$. Therefore, using that $G$ is abelian 
    \[
    ab = x_1^2x_2^2 = x_1x_1x_2x_2 = x_1x_2x_1x_2 = (x_1x_2)^2    
    \]
    and thus $ab \in G$ by closure of groups under multiplication and $H$ is a subgroup of $G$. 
\end{proof}
\subsection*{c.}
\begin{proof}
    Let $f:G \to H$ be a map where $x \mapsto x^2$ for $x \in G$. First to show this is a homomorphism look at $a, b \in G$, and since $G$ is abelian  
    \[
    f(ab) = (ab)^2 = abab = aabb = a^2b^2 = f(a)f(b).
    \]
    Now, since $f$ is a homorphism, clearly $H = \{x^2 : x \in H\}$ is the image but the kernel is where $f(x) = e$ and this is precisely when $f(x) =x^2 = e$, thus these are elements of order $2$ and the identity element. The kernel is thus exacly $K$. Then, by the first isomorphism theorem we have that \[
      G/K \cong H.   
    \]
\end{proof}

\section*{5.}
\begin{proof}
    Suppose $f:G \to H$ is a homomorphism of finite groups. Then, we claim the image $\{f(g) : g \in G\}$ is a subgroup of $H$. Suppose $a \in Im(f)$, then there exists $x \in G$ such that $f(x) = a$, and since $f(x^{-1}) = f(x)^{-1}$ for a homomorphism, then $a^{-1} = f(x)^{-1} = f(x^{-1})$ and since $x^{-1} \in G$, $f(x^{-1}) = a^{-1} \in Im(f)$. Next, if $a, b \in Im(f)$ we have that there exists $x_1, x_2 \in G$ such that $f(x_1) = a$ and $f(x_2) = b$, so since $f$ is a homomorphism
    \[
    ab = f(x_1)f(x_2) = f(x_1x_2)    
    \]
    and $x_1x_2 \in G$, so $f(x_1x_2) = ab \in Im(f)$. 
    Therefore, $Im(f) \leq H$ is a subgroup of $H$ and by Lagrange's theorem $|Im(f)|$ divides $|H|$. By the First Isomorphism theorem 
    \[
    G/ker(f) \cong Im(f)  
    \]
    and we know that by Lagrange 
    \[
    \frac{|G|}{|ker(f)|} = [G: ker(f)]    
    \]
    but $[G: ker(f)]$ is the number of right cosets of $ker(f)$ which is exactly the order of $G/ker(f)$ which is isomorphic to $Im(f)$. Thus, 
    \[
    |G| = [G: ker(f)]|ker(f)| = |G/ker(f)||ker(f)| = |Im(f)||ker(f)|    
    \]
    implying $|Im(f)|$ divides $|G|$. k
\end{proof}

\section*{6.}
\subsection*{a.}
\begin{proof}
To prove that $N$ is a normal subgroup take $x \in NK$ and $n \in N$. Because $x \in NK$, $x = n_1k_1$ and $x^{-1} = (n_1k_1)^{-1}$. Because $N$ is normal in $G$, $n_1k_1 = k_1n_2$ for some $n_2 \in N$. Then, using the normality of $N$ in $G$ we have $k_1N = Nk_1$, so  \[
xN = = n_1k_1N = k_1n_2N = k_1N = Nk_1 = Nn_1k_1 = Nx.\] 

\end{proof}
\subsection*{b.}
\begin{proof}
    We begin by showing $f$ is a homomorphism, so take $k_1k_2 \in K$. Note that $N$ is normal in $NK$, so $Nk_1 = k_1N$. Therfore, 
    \[
    f(k_1k_2) = Nk_1k_2 = NNk_1k_2 = Nk_1Nk_2 = f(k_1)f(k_2).    
    \]
    To show the homomorphism is surjective, suppose we have $Na \in NK/N$ for $a = n_1k_1 \in NK$ and $n_1 \in N$, $k_1 \in K$. Then, 
    \[
    Na = Nn_1k_1 = Nk_1    
    \]
    so $\exists k_1 \in K$ such that 
    \[
    f(k_1) = Nk_1 = Nn_1k_1 = Na    
    \]
    and thus the homomorphism is surjective. If $f(k) = N$, then $f(k) = Nk = N$ which implies $k \in N$. Thus, the kernel must be $N \cap K$, as $k \in K$ must also be in $N$ for $f(k) = N$. 
\end{proof}
\subsection*{c.}
\begin{proof}
    We have met all the criteria to apply the First Isomorphism theorem and state
   \[
   K/(N \cap K) \cong NK / N. 
   \] 
\end{proof}
\end{document}

