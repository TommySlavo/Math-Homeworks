\documentclass{article}
\usepackage[utf8]{inputenc}
\usepackage{mathrsfs}
\usepackage{tikz}
\usepackage{amssymb}
\usepackage{amsthm}
\usepackage{graphicx} % Required for inserting images
\usepackage{amsmath}
\usepackage{MnSymbol}
\usepackage{geometry}
\usepackage{bbm}
\usepackage{physics}
\usepackage{verbatim}
\usepackage{enumerate}
\allowdisplaybreaks
\newcommand{\mycomment}[1]{}
\newcommand\numeq[1]%
  {\stackrel{\scriptscriptstyle(\mkern-1.5mu#1\mkern-1.5mu)}{=}}
\newcommand\numleq[1]
  {\stackrel{\scriptscriptstyle(\mkern-1.5mu#1\mkern-1.5mu)}{\leq}}
\newcommand\numgeq[1]
  {\stackrel{\scriptscriptstyle(\mkern-1.5mu#1\mkern-1.5mu)}{\geq}}



\title{Math 110B Homework 8}
\author{Tom Slavonia}
\date{\today}

\begin{document}
\maketitle

\section*{1.}
\begin{proof}
    Consider the Sylow $3-$subgroup of $S_4$ $\langle (123) \rangle$. We need to find the normalizers of the subgroup in order to have that the conjugacy class is preverved using the definition of conjugacy class. Note that if we have $x \in G$ such that the conjugacy class is preserved, by the second Sylow theorem we have that all Sylow $p-$subgroups are conjugate to one another. Therefore, every other Sylow $3-$subgroup is conjugate to this Sylow $3-$subgroup. The identity element can be trivially included in the conjugates as well. 
\end{proof}

\section*{2.}
\begin{proof}
    For $C_a = \{gag^{-1}: g \in G\}$ take $b \in C_a$. We then have $b = gag^{-1}$ for some $g \in G$. Therefore, 
    $f(b) = f(gag^{-1}) = f(g)f(a)f(g^{-1}) = f(g)f(a)f(g)^{-1} \in f(C_a)$ and thus $f(C_a)$ is also a conjugacy class of $G$. 
\end{proof}

\section*{3.}
\begin{proof}
    Let $G$ be an infinite group and $H\subset G$ be all elements that have finite conjugates. Consider $a, b \in H$. Then, if $a$ in $H$ implies that $a $ has finite conjugates and therefore every conjugate of $a$ can be written $gag^{-1}$, so every conjugate of $a^{-1}$ is $(gag^{-1})^{-1} = ga^{-1}g$ and therefore $a^{-1} \in H$. The product $ab$ has conjugates $gabg^{-1} = gag^{-1}gbg^{-1} = (gag^{-1})(gbg^{-1})$ and $a, b$ have finite conjugages, so $ab$ has finite conjugate as we can write it in terms of conjugates of $a$ and $b$. Thus, $H$ is closed under inverses and multiplication and is thus a subgroup.  
\end{proof}

\section*{4.}
\begin{proof}
    Let $H$ be a proper subgroup of $G$. By theorem 9.25 we know that the number of $H$ conjugates of $G$ must divide the order of $G$. Therefore, if $H$ is a proper subset of $G$, then since $H$ is a normal subgroup of $N(H)$, then $[H : H \cap N(H)]$ is strictly less than the order of $G$ implying $G$ cannot be the union of all the conjugates of $H$. 
\end{proof}

\section*{5.}
\begin{proof}
    To begin, by viewing the multiplication table we can see that there are exactly 2 generators, same as in $D_4$, then we can see that all elements correspond to multiplication just as the same as in $D_4$. Similarly the claim holds for $G_2$ and $Q_8$. Hence, we get the desired result. 
\end{proof}

\section*{6.}
\begin{proof}
    Take $(n_1, k_1), (n_2, k_2), (n_3, k_3) \in N\rtimes_{\phi} K$. We then have
    \begin{align*}
        ((n_1, k_1)(n_2, k_2))(n_3, k_3) &= ((n_1\phi_{k_1}(n_2), k_1k_2))(n_3, k_3) \\
        &= (n_2\phi_{k_1}(n_2), k_1k_2)(n_3, k_3) \\
        &= (n_3\phi_{k_1k_2}(n_3)\phi_{k_1}(n_2), k_1k_2k_3)\\
        &= (n_1\phi_{k_1}(n_2\phi_{k_2}(n_3)),k_1k_2k_3)\\
        &= (n_1, k_1)(n_2\phi_{k_2}(n_3), k_2k_3)\\
        &= (n_1, k_1)((n_2\phi_{k_2}(n_3), k_2k_3)) \\
        &= (n_1, k_1)((n_2, k_2)(n_3, k_3))
    \end{align*}
    using that automorphisms are themselves group homomorphisms. Thus, we have the associative property and the semidirect product is a group. 
\end{proof}
\end{document}