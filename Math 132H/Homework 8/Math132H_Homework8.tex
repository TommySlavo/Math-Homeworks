\documentclass{article}
\usepackage[utf8]{inputenc}
\usepackage{mathrsfs}
\usepackage{tikz}
\usepackage{amssymb}
\usepackage{amsthm}
\usepackage{graphicx} % Required for inserting images
\usepackage{amsmath}
\usepackage{bbm}
\usepackage{MnSymbol}
\usepackage{geometry}
\usepackage{physics}
\usepackage{verbatim}
\usepackage{enumerate}
\allowdisplaybreaks
\newcommand\numeq[1]%
  {\stackrel{\scriptscriptstyle(\mkern-1.5mu#1\mkern-1.5mu)}{=}}
\newcommand\numleq[1]
  {\stackrel{\scriptscriptstyle(\mkern-1.5mu#1\mkern-1.5mu)}{\leq}}
\newcommand\numgeq[1]
  {\stackrel{\scriptscriptstyle(\mkern-1.5mu#1\mkern-1.5mu)}{\geq}}



\title{Math 132H Homework 8}
\author{Tom Slavonia}
\date{\today}

\begin{document}
\maketitle

\section*{10.}
\begin{proof}
   If we consider $G:\mathbb{H} \to \mathbb{D}$ where 
   \[
   z \mapsto \frac{z - i}{z + i}. 
   \] 
   We then have that $|G(z)| \leq 1$ always, and any function that maps from $\mathbb{H} \to \mathbb{C}$ such that $|F(z)|\leq 1$ it must be that $F$ maps to some subset of the closed unit disc. Also, with $G$ we have that $G(i) = 0$. We have that $G$ and $G^{-1}$ are conformal maps, and the composition of two holomorphic maps is holomorphic. Therefore, $F\circ G^{-1}: \mathbb{H} \to \mathbb{C}$ is a holomorphic map such that $F(i) = 0$ and by Schwarz lemma we have 
   \[
   |F(G^{-1}(z))| \leq z \leq 1 \text{ as } z \in \mathbb{D}.  
   \]
   Thus, we have that 
   \[
   |F(G^{-1}(G(z)))| = |F(z)| \leq |G(z)| = \left|\frac{z - i}{z + i}\right|.
   \]


\end{proof}

\section*{11.}
\begin{proof}
   Consider the maps $\varphi:D(0, R) \to \mathbb{D}$ and $\phi:D(0, M) \to \mathbb{D}$ where $z\mapsto \frac{z}{R}$ and $z \mapsto \frac{z}{M}$. We then have that $\varphi^{-1}:\mathbb{D} \to D(0, R)$ maps $z \mapsto Rz$ and thus we define $F = \phi \circ f \circ \varphi^{-1}$. Therefore, we have
   \[
   F(z) = \phi(f(zR)) = \frac{f(zR)}{M}. 
   \]
   Note that with $z \in \mathbb{D}$ we have $zR \in D(0, R)$. 
   Then, 
   \begin{align*}
   \left|\frac{F(z) - F(0)}{1 - \overline{F(0)}F(z)}\right| &= \left|\frac{\frac{f(zR)}{M} - \frac{f(0)}{M}}{1 - \frac{\overline{f(0)}}{M}\frac{f(zR)}{M}}\right| \\
   &= \left|\frac{\frac{f(zR) - f(0)}{M}}{\frac{M^2- \overline{f(0)}f(zR)}{M^2}}\right| \\
   &= \left|\frac{Mf(zR) - Mf(0)}{M^2 - \overline{f(0)}f(zR)}\right| \leq z
   \end{align*}
   with the last step being justified by the Schwartz lemma. Rearranging we get 
   \[
    \left|\frac{f(zR) - f(0)}{M^2 - \overline{f(0)}f(zR)}\right| \leq \frac{z}{M}.
   \]
   Then if we set $w = zR \in D(0, R)$ we obtain
   \[
   \left|\frac{f(w) - f(0)}{M^2 - \overline{f(0)}f(w)} \right| \leq \frac{w}{MR} 
   \]
   and the result is proven. 
\end{proof}

\section*{12.}
\subsection*{a.}
\begin{proof}
  Suppose $f:\mathbb{D}\to \mathbb{D}$ is analytic with two fixed points. Suppose $\alpha, \alpha'$ are the two fixed points. Consider the map $\varphi_{\alpha}:\mathbb{D}\to \mathbb{D}$ such that for $|\alpha| < 1$
  \[
  \varphi_{\alpha}(z) = \frac{\alpha - z}{1 - \overline{\alpha}z}  
  \]
  and since $|\alpha| < 1$ the map will never have a pole but is equal to $0$ when $z = \alpha$. The book proves that $\varphi_{\alpha}$ is indeed an automorphism of $\mathbb{D}$. With the choice of $\varphi_{\alpha}$ as proven in the book we also have $\varphi_{\alpha}^{-1} = \varphi_{\alpha}$. Define 
  \[
  \tilde{f} = \varphi_{\alpha}\circ f \circ \varphi_{\alpha}  
  \]
  and we clearly have 
  \[
  \tilde{f}(0) = \varphi_{\alpha}(f(\varphi_{\alpha}(0))) = \varphi_{\alpha}(f(\alpha)) = \varphi_{\alpha}(\alpha) = 0  
  \]
  using that $\varphi_{\alpha}(0) = \alpha$, $f(\alpha) = \alpha$ as it is a fixed point of $f$, and $\varphi_{\alpha}(\alpha) = 0$. But, we also obtain a second fixed point for $\tilde{f}$, namely $\varphi_{\alpha}(\alpha')$:
  \[
  \varphi_{\alpha}(f(\varphi_{\alpha}(\varphi_{\alpha}(\alpha')))) = \varphi_{\alpha}(f(\alpha')) = \varphi_{\alpha}(\alpha')  
  \] as $\varphi_{\alpha}\circ\varphi_{\alpha}(z) = z$ and $\alpha'$ is a fixed point of $f$ by assumption. Hence $\tilde{f}$ has two fixed points, but since $\tilde{f}(0) = 0$ and clearly $\tilde{f}:\mathbb{D}\to \mathbb{D}$ is holomorphic as $\varphi_{\alpha}$ and $f$ are holomorphic, we may apply the Schwartz lemma. Since we have a nonzero $z_0 = \varphi_{\alpha}(\alpha')$ such that 
  \[
  |\tilde{f}(\varphi_{\alpha}(\alpha'))| = |\varphi_{\alpha}(\alpha')|  
  \]
  by Schwartz's lemma it must be that $\tilde{f}$ is a rotation. But it can't be a rotation that isn't the identity; otherwise, every point will be rotated by the same amount, meaning we can't have two fixed points. Thus, $\tilde{f} = id$ and it must be that $f = id$. 
\end{proof}

\subsection*{b.}
\begin{proof}
  We have already looked at the maps $G:\mathbb{D} \to \mathbb{H}$ where $G(z) = i \frac{1 - z}{1 + z}$ and $F:\mathbb{H}\to \mathbb{D}$ with $F(w) = \frac{i - w}{i + w}$. Therefore, composing these conformal maps with the conformal map $f:\mathbb{H} \to \mathbb{H}$ with $f(z) = z + 1$ we have $F\circ f\circ G: \mathbb{D} \to \mathbb{D}$ a conformal map from the unit disc to itself, but with no fixed points as in the upper half plane every point is shifted by $1$ and then remapped into the disc. 
\end{proof}

\section*{13.}
\subsection*{a.}
\begin{proof}
  We can begin by defining $\psi_{\alpha}: \mathbb{D} \to \mathbb{D}$ by $\psi_{\alpha}(z) = \frac{z - \alpha}{1 - \overline{\alpha}z}$. Note that the book has already discussed that this is an automorphism of the unit disc. The inverse that arises from the definition of $\psi_{\alpha}$ is $\psi_{\alpha}^{-1}(z) = \frac{z + \alpha}{1 + z\overline{\alpha}}$ which is once again an automorphism of the unit disc. Therefore, $\psi_{f(w)}\circ f \circ \psi_{w}^{-1}$ is a holomorphic function from the unit disc to itself as it is the composition of holomorphic funcitions from the unit disc to itself. Note that we then have 
  \[
  \psi_{f(w)}(f(\psi_w^{-1}(0)))  = \psi_{f(w)}(f(w)) = \frac{f(w) - f(w)}{1 - |f(w)|^2} = 0 
  \]
  as $|f(w)| < 1$. Thus, we meet the necessary conditions to apply Schartz's lemma and attain the result
  \[
  |\psi_{f(w)}(f(\psi_w^{-1}(z)))| \leq |z|  
  \]
  and setting $z = \psi_w(z)$ we have that 
  \[
    |\psi_{f(w)}(f(\psi_w^{-1}(\psi_w(z))))| = |\psi_{f(w)}(f(z))| \leq |\psi_w(z)|
  \]
  thus giving us the result
  \[
   \rho(f(z), f(w)) \leq \rho(z, w). 
  \]
  If $f$ is an automorphism of $\mathbb{D}$, then $f^{-1}$ exists as a conformal map from $\mathbb{D} \to \mathbb{D}$. Now consider 
  \[
    \psi_{f^{-1}(w)}\circ f^{-1}\circ \psi_w^{-1}(0) = \psi_{f^{-1}(w)}(f^{-1}(w)) = \frac{f^{-1}(w) - f^{-1}(w)}{1 - |f^{-1}(w)|^2} = 0
  \]
  as $f^{-1}(w) \in \mathbb{D}$. Clearly the composition of automorphisms of the disc above is holomorphic and therefore the conditions of Schwarz lemma are satisfied and we have 
  \[
  |\psi_{f^{-1}(w)}(f^{-1}(\psi_{w}^{-1}(z)))| \leq |z|.
  \]
  Set $z = \psi_w(z)$ and we ahve 
  \[
  |\psi_{f^{-1}(w)}(f^{-1}(z))| \leq |\psi_w(z)|  
  \]
  and once again we can set $z = f(z)$ which implies $w = f(w)$ and see 
  \[
  |\psi_w(z)| \leq |\psi_{f(w)}(f(z))|  
  \]
  and thus 
  \[
  \rho(z, w) \leq \rho(f(z), f(w)).   
  \]
  Given that $f$ is an automorphism we have 
  \[
    \rho(z, w) \leq \rho(f(z), f(w)) \text{ and } \rho(f(z), f(w)) \leq \rho(z, w)
  \]
  implying by squeeze lemma 
  \[
  \rho(z, w) = \rho(f(z), f(w)).  
  \]
\end{proof}
\subsection*{b.}
\begin{proof}
  From the previous result we know 
  \[
  \left|\frac{f(z) - f(w)}{1 - \overline{f(w)}f(z)}\right| \leq \left|\frac{z - w}{1 - \overline{w}z}\right|  
  \]
  with $z , w \in \mathbb{D}$. Rearranging we gather
  \[
  \left|\frac{f(z) - f(w)}{(z - w)(1 - \overline{f(w)}f(z))}\right| \leq \left|\frac{1}{1 - \overline{w}z}\right|.  
  \]
  Taking the limit as $z \to w$: 
  \[
  \left|\lim\limits_{z \to w} \frac{f(z) - f(w)}{z - w} \cdot \frac{1}{1 - \overline{f(w)}f(z)} \right| \leq \frac{1}{|1 - |w|^2|}, 
  \]
  as $f$ is holomorphic we may more the limit inside $f$, therefore
  \[
  \left|\frac{f'(w)}{1 - |f(w)|^2} \right| \leq \frac{1}{|1 - |w|^2|}  
  \]
  and 
  \[
  \frac{|f'(w)|}{1 - |f(w)|^2} \leq \frac{1}{1 - |w|^2}  
  \]
  with 
  \[
  |1 - |f(w)|^2| = 1 - |f(w)|^2 \text{ and } |1 - |w|^2| = 1 - |w|^2  
  \]
  because $w, f(w) \in \mathbb{D}$. 
\end{proof}

\section*{14.}
\begin{proof}
  Suppose $f:\mathbb{H} \to \mathbb{D}$ is a conformal map. 
 Recall we have the conformal map $g: \mathbb{D} \to \mathbb{H}$ by 
 \[
 g(z) = i \frac{1 - z}{1 + z}. 
 \] 
 Hence, $f \circ g$ is a conformal map and thus an automorphism of $\mathbb{D}$. Then we have for $z \in \mathbb{D}$
 \[
 f(g(z)) = f\left(i \frac{1 - z}{1 + z} \right). 
 \]
 By a theorem from the book it must be that any automorphism of the disc is of the form 
 \[
 e^{i \theta} \frac{\alpha - z}{1 - \overline{\alpha}z} 
 \]
 for $\alpha\in \mathbb{D}$ and $\theta \in \mathbb{R}$, so 
 \[
  f(g(z)) = f\left(i \frac{1 - z}{1 + z} \right) = e^{i \theta} \frac{\alpha - z}{1 - \overline{\alpha}z}.
 \]
 The motivation of the next step comes from noting that as previously mentioned in the book
 \[
 g^{-1}(z) = \frac{i - z}{i + z}. 
 \]
 Take $z = \frac{i - z}{i + z}$ and we have 
 \begin{align*}
  f(z) &= e^{i \theta} \frac{\alpha - \frac{i - z}{i + z}}{1 - \overline{\alpha} \frac{i - z}{i + z}} \\
  &= e^{i \theta} \frac{\alpha i + \alpha z - i + z}{i + z - \overline{\alpha}i + \overline{\alpha}z} \\
  &= e^{i \theta} \frac{z(1 + \alpha) + \alpha i - i}{z(1 + \overline{\alpha}) + i -\overline{\alpha}i} \\
  &= e^{i \theta} \frac{z + \frac{\alpha i - i }{1 + \alpha}}{z + \frac{i - \overline{\alpha}i}{1 + \overline{\alpha}}}\\
  &= e^{i \theta}\frac{z - \frac{i - \alpha i}{1 + \alpha}}{z - \frac{\overline{\alpha}i - i}{1 + \overline{\alpha}}}\\
  &= e^{i \theta} \frac{z - i\frac{1 - \alpha}{1 + \alpha}}{z - i \frac{\overline{\alpha} - 1}{1 + \overline{\alpha}}}.
 \end{align*}
 Notice 
 \[
 \overline{i\frac{1 - \alpha }{1 + \alpha}} = i\frac{\overline{\alpha}- 1}{1 + \overline{\alpha}} 
 \]
and that by the known mapping $g(z) = i\frac{1 - z}{1 + z} \in \mathbb{H}$ and with $\alpha \in \mathbb{D}$ we have 
\[
   i\frac{1 - \alpha}{1 + \alpha} \in \mathbb{H}.
\]
 Taking 
 \[
 \beta =  i\frac{1 - \alpha }{1 + \alpha} \text{ and then } \overline{\beta} = i\frac{\overline{\alpha} - 1}{1 + \overline{\alpha}} 
 \]
 we get the result 
 \[
 f(z) = e^{i \theta}\frac{z - i\frac{1 - \alpha }{1 + \alpha}}{z - i\frac{\overline{\alpha} - 1}{1 + \overline{\alpha}}} = e^{i \theta}\frac{z - \beta}{z - \overline{\beta}}.
 \]
\end{proof}

\section*{15.}
\subsection*{a.}
\begin{proof}
  We know by previous results that all automorphisms of the upper half-plane are of the form 
  \[
  z \mapsto \frac{az + b}{cz + d}.  
  \]
  The identity map fixes all points so it certainly fixes three different points. Suppose $z$ is a fixed point by an automorphism of $\mathbb{H}$, then 
  \[
  z = \frac{az + b}{cz + d}.  
  \]
  Rearranging the above implies 
  \[
  \frac{az + b - cz^2 - dz}{cz + d} = 0  
  \]
  and thus that 
  \[
  -z^2 + z(a - d) + b = 0.  
  \]
  According to the Fundamental theorem of Algebra, the polynomial above has two solutions, so three points can't be fixed unless the map is the identity.  
\end{proof}

\subsection*{b.}
\begin{proof}
  The idea here is that we build a map that assigns $x_1$, $x_2$, and $x_3$ to three manageable points. Define
  \[
  f(z) = \frac{(z - x_1)(x_2 - x_3)}{(z - x_3)(x_2 - x_1)}  
  \]
  here $f(x_1) = 0$, $f(x_2) = 1$, and $\lim\limits_{z \to x_3}f(z) = \infty$. This is an automorphism of $\mathbb{H}$, as the determinant of the coefficients is greater than $0$ which is sufficient as $z \in \mathbb{H}$. Also, define 
  \[
  g(z) = \frac{zy_2y_3 - zy_1y_3 - y_1y_2 + y_1y_3}{zy_2 - zy_1 - y_2 + y_3}  
  \] 
  which is the inverse of 
  \[
  \frac{(z - y_1)(y_2 - y_3)}{(z - y_3)(y_2 - y_1)},  
  \]
  but I will spare the reader that long and tedious computation. 
  We then have that 
  \[
  g \circ f(x_1) = \frac{-y_1y_2 + y_1y_3}{-y_2 + y_3} = y_1  
  \]
  \[
  g \circ f(x_2) = \frac{y_2y_3 - y_1y_2}{y_3 - y_1} = y_2  
  \]
  \[
  \lim\limits_{z \to x_3}g\circ f(z) = \frac{\infty\cdot y_2y_3 - \infty\cdot y_1 y_3}{\infty \cdot y_2 - \infty \cdot y_1} = y_3.  
  \]

  Suppose $f$ and $g$ are two conformal maps such that $f(x_1) = g(x_1) = y_1$, $f(x_2) = g(x_2) = y_2$ and $f(x_3) = g(x_3) = x_3$ with all variables as defined by the problem. By the first part of the problem, it must be that $f \circ g^{-1} = id$ as the map would fix three points, and thus $f = g$. 
\end{proof}

\section*{17.}
\begin{proof}
  We begin by using the hint provided for the first integral, since $\psi_{\alpha}$ is holomorphic and bijective, we may use that integrals of the form 
  \[
    \frac{1}{\pi}\iint_{\mathbb{D}}|\psi_{\alpha}'|^2dx dy = \frac{1}{\pi}\cdot \text{Area}(\psi_{\alpha}(\mathbb{D}))
  \]
  but $\psi_{\alpha}$ is a automorphism of $\mathbb{D}$, so
  \[
  \psi_{\alpha}(\mathbb{D}) = \mathbb{D}  
  \]
  and since the radius of the unit disc is $1$
  \[
    \frac{1}{\pi}\iint_{\mathbb{D}}|\psi_{\alpha}'|^2dx dy = \frac{1}{\pi}\cdot \text{Area}(\psi_{\alpha}(\mathbb{D})) = \frac{1}{\pi}\cdot \text{Area}(\mathbb{D}) = \frac{1}{\pi}\cdot \pi = 1.
  \]\\

  The second integral must be handled with more care. If we parametrize $z = e^{i \theta}$. Now, we can calculate the derivative of 
  \[
  \psi_{\alpha}(z) = \frac{\alpha - z}{1 - \overline{\alpha}z}.  
  \]
  Using the quotient rule we calculate the derivative to be
  \[
   \psi_{\alpha}'(z) = \frac{-1 +|\alpha|^2}{(1 - \overline{\alpha}z)^2}.
    \]
    Plugging into the integral 
    \[
    \frac{1}{\pi}\iint_{\mathbb{D}} |\psi_{\alpha}'|dxdy = \frac{1}{\pi}\iint_{\mathbb{D}} \left|\frac{-1 + |\alpha|^2}{(1 - \overline{\alpha}z)^2} \right|dxdy.
    \]
    With our parametrization of $z$ in hand, we will tackle this integral over contour of the unit disc. Note that since $\alpha \in \mathbb{D}$ we have
    \[
    |1 - |\alpha|^2| = 1 - |\alpha|^2.  
    \]
    Therefore, with $d \theta = \frac{1}{iz}dz$ our integral becomes
    \begin{align*}
      \frac{1}{\pi}\iint_{\mathbb{D}} |\psi_{\alpha}'|dxdy &= \frac{1 - |\alpha|^2}{\pi}\int_0^1\int_{0}^{2\pi} \frac{1}{|1 - \overline{\alpha}re^{i \theta}|^2}r d\theta dr  \\
      &= \frac{1 - |\alpha|^2}{\pi}\int_0^1\int_{|z| = 1} \frac{r}{iz|1 - \overline{\alpha} rz|^2}dz dr.
    \end{align*}
    With $|z| = 1$ we have $z\cdot \overline{z} = 1$ and so $\overline{z} = \frac{1}{z}$ and the integral is now
     \begin{align*}
      \frac{1}{\pi}\iint_{\mathbb{D}}|\psi_{\alpha}'|dxdy &= \frac{1 - |\alpha|^2}{\pi}\int_0^1\int_{|z| = 1}\frac{r}{iz|1 - \overline{\alpha} rz|^2}dzdr \\
      &= \frac{1 - |\alpha|^2}{\pi}\int_0^1\int_{|z| = 1} \frac{r}{iz(1 - \overline{\alpha} r z)\left(1 - \alpha r \frac{1}{z}\right)} dz dr \\
      &= \frac{1 - |\alpha|^2}{i\pi}\int_0^1\int_{|z| = 1} \frac{r}{z(1 - \overline{\alpha} rz)\left(\frac{z - \alpha r}{z}\right)}dz dr \\
      &= \frac{1 - |\alpha|^2}{i\pi}\int_0^1\int_{|z| = 1} \frac{rz}{z(1 - \overline{\alpha} rz)\left(z - \alpha r\right)}dz dr \\
      &= \frac{1 - |\alpha|^2}{i\pi}\int_0^1\int_{|z| = 1} \frac{r}{(1 - \overline{\alpha} rz)\left(z - \alpha r\right)}dz dr \\
      &= \frac{1 - |\alpha|^2}{i \pi}\int_0^1 r \int_{|z| = 1} \frac{1}{(1 - \overline{\alpha} r z)(z - \alpha r)}dzdr.
     \end{align*}
     Now we must find the residues of the poles. The poles are $z = \frac{1}{\alpha r}$ and $z = \alpha r$. Note that since $r < 1$ and $\alpha \in \mathbb{D}$ we have that the pole $ \frac{1}{\alpha r}$ is not in the disc therefore we need only consider the pole $z_1 = \alpha r$. Clearly, this pole only has an order of $1$. Calculating the residue we get
     \[
     res_{z_1}f = \lim\limits_{z \to \alpha r}( z - \alpha r) \frac{1}{(1 - \overline{\alpha} rz)(z - \alpha r)} = \frac{1}{1 - |\alpha|^2r^2}. 
     \]
     Thus, by the residue integration formula
     \begin{align*}
      \frac{1}{\pi}\iint_{\mathbb{D}}|\psi_{\alpha}'|dxdy &= \frac{1 - |\alpha|^2}{i \pi}\int_0^1 r \int_{|z| = 1} \frac{1}{(1 - \overline{\alpha} r z)(z - \alpha r)}dzdr \\
      &= \frac{1 - |\alpha|^2}{i\pi} \int_0^1 r \frac{2\pi i }{1 - |\alpha|^2r^2}dr \\
      &= (2 - 2|\alpha|^2)\int_0^1 \frac{r}{1 - |\alpha|^2r^2}dr.
     \end{align*}
     Performing classical $u-$substitution with $u = 1 - |\alpha|^2r^2$ we have 
     \[
     \frac{du}{dr} = -2r|\alpha|^2
     \]
     and thus
     \[
     -\frac{du}{2r|\alpha|^2} = dr. 
     \]
     We can then simplify the integral into 
     \begin{align*}
      \frac{1}{\pi}\iint_{\mathbb{D}}|\psi_{\alpha}'|dxdy &= (2 - 2|\alpha|^2)\int_0^1 \frac{r}{1 - |\alpha|^2r^2}dr \\
      &= (2 - 2|\alpha|^2)\int_1^{1 - |\alpha|^2}\frac{r}{u(-2r|\alpha|^2)}du \\
      &= -\frac{1 - |\alpha|^2}{|\alpha|^2}\int_1^{1 - |\alpha|^2} \frac{1}{u}du \\
      &= \frac{1 - |\alpha|^2}{|\alpha|^2} \int_{1 - |\alpha|^2}^1 \frac{1}{u}du \\
      &= -\frac{1 - |\alpha|^2}{|\alpha|^2}\left[\ln(|u|)\right]_{1 - |\alpha|^2}^{1} \\
      &= \frac{1 - |\alpha|^2}{|\alpha|^2}(-\ln(|1 - |\alpha|^2|)) \\
      &= \frac{1 - |\alpha|^2}{|\alpha|^2}\ln((1 - |\alpha|^2)^{-1}) \\
      &= \frac{1 - |\alpha|^2}{|\alpha|^2}\ln\left(\frac{1}{1 - |\alpha|^2}\right)
     \end{align*}
     with the last step true by basic logarithm rules. 
     Hence, the last proof of my undergraduate career is complete. 
    
\end{proof}


\end{document}