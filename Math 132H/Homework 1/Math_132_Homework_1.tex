\documentclass{article}

\usepackage[utf8]{inputenc}
\usepackage{mathrsfs}
\usepackage{tikz}
\usepackage{amssymb}
\usepackage{amsthm}
\usepackage{graphicx} % Required for inserting images
\usepackage{amsmath}
\usepackage{MnSymbol}
\usepackage{geometry}
\usepackage{physics}
\usepackage{enumerate}
\usepackage{enumitem}
\newcommand\numeq[1]%
  {\stackrel{\scriptscriptstyle(\mkern-1.5mu#1\mkern-1.5mu)}{=}}
\geometry{a4paper, margin=1in} % set margins
\title{Math 132H Homework 1}
\author{Thomas Slavonia; UID: 205511702}
\date{April 2024}

\begin{document}

\maketitle

\section*{2.}
\begin{proof}
Let $z, w \in \mathbb{C}$ where $z = x_1 + iy_1$ and $w = x_2 + iy_2$. Now, expanding the relevant equation, we get:
\begin{align*}(z, w) + (w, z) &= z\bar{w} + w\bar{z} \\
    &= (x_1 + iy_1)(x_2 -iy_2) + (x_2 + iy_2)(x_1 - iy_1) \\
    &= x_1x_2 -ix_1y_2 + ix_2y + y_1y_2 + x_1x_2 -ix_2y_1 + ix_1y_2 + y_1y_2 \\
    &= 2x_1x_2 + 2y_1y_2 \\
    &= 2 \langle z, w \rangle. \end{align*}
Thus, we can conclude that 
\[
    \frac{(z, w) + (w, z)}{2} = \langle z, w \rangle = \Re(z, w).
    \]

\end{proof}



\section*{7.}
\subsection*{a.}
\begin{proof}
Let $z = r_1e^{i \theta_1}$ and $w = r_2e^{i \theta_2}$. Using the fact that $\left|e^{i \theta_1}\right| = 1$, 
\begin{align*}
    \left|\frac{w - z}{1 - \bar{w}z} \right| &= \left|\frac{r_2e^{i \theta_2} - r_1e^{i \theta_1}}{1 - r_1r_2e^{i(-\theta_2 + \theta_1)}} \right| \\
    &= \left|e^{i \theta_1}\frac{r_2e^{i(\theta_2 - \theta_1)} - r_1}{1 - r_1r_2e^{i(-\theta_2 + \theta_1)}}\right|\\
    &= \left|e^{i \theta_1} \right|\left|\frac{r_2e^{i(\theta_2 - \theta_1)} - r+1}{1 - r_1r_2e^{i(-\theta_2 + \theta_1)}} \right| \\
    &=\left|\frac{r_2e^{i(\theta_2 - \theta_1)} - r+1}{1 - r_1r_2e^{i(-\theta_2 + \theta_1)}} \right|. 
\end{align*}
Define $w' = r_2e^{i(\theta_2 - \theta_1)}$ and note that \[|w'| = |r_2e^{i(\theta_2 - \theta_1)}| = r_2|e^{i(\theta_2 - \theta_1)}| = r_2 = |w| < 1. \]
Therefore, we can now look at the equation 
\[
    \left|\frac{w' - r_1}{1 - r_1\bar{w}'} \right|
\]
where now $z \in \mathbb{R}$. Now, by the hint given in the textbook, we have to show that \[
    (r_1 - w)(r_1 - \bar{w}) \leq (1 - r_1w)(1 - r_1\bar{w})
    \] is true. Thus, 
    \begin{align*}
        & (r_1 - w)(r_1 - \bar{w}) \leq (1 - r_1w)(1 - r_1\bar{w}) \\
        &\iff r_1^2 - r_1\bar{w} - r_1w + |w|^2 \leq 1 - r_1\bar{w} - r_1w +r_1^2|w|^2 \\
        &\iff r_1^2 + |w|^2 \leq 1 + r_1^2|w|^2 \\
        &\iff |w|^2(1 - r_1^2) \leq 1 - r_1^2
    \end{align*}
    which holds because $|w|^2 < 1$ and $r_1 < 1$. 

\end{proof}
\subsection*{b.}
\subsubsection*{i)}
\begin{proof}
    Fix $w \in \mathbb{D}$ the unit disk. Take $z \in \mathbb{D}$. We want to show that $F(z) \in \mathbb{D}$. 
    Since both $z, w \in \mathbb{D}$ we have that $|z|, |w| < 1$ and thus, by the previous part of the problem, we have that $\left|\frac{w - z}{1 -\bar{w}z} \right| < 1$ which implies that $F(z) =\left|\frac{w - z}{1 -\bar{w}z} \right| \in \mathbb{D}$, so $F: \mathbb{D} \to \mathbb{D}$. \\
    To prove $F$ is holomorphic, note that $F(z) = \frac{w - z}{1 - \bar{w}z}$ is of the form $\frac{f}{g}$ where $f(z) = w - z$ and $g(z ) = 1 - \bar{w}z$. First, we will show that $f(z)$ is holomorphic:
    \begin{align*}
        f'(z)&= \lim\limits_{h \to 0} \frac{f(z + h) - f(z)}{h} \\
        & = \lim\limits_{h \to 0} \frac{w - z - h - w + z}{h} \\
        &= \lim\limits_{h \to 0} \frac{- h}{h} \\
        &= -1. 
    \end{align*}
    Thus, $f(z)$ is holomorphic. Now we will show that $g(z)$ is holomorphic:
    \begin{align*}
        g'(z) &= \lim\limits_{h \to 0} \frac{g(z + h) - g(z)}{h} \\
        &= \lim\limits_{h \to 0} \frac{1 - \bar{w}(z + h) - 1 + \bar{w}z}{h} \\
        &= \lim\limits_{h \to 0} \frac{1 - \bar{w}z - \bar{w}h - 1 + \bar{w}z}{h} \\
        & = \lim\limits_{h \to 0} \frac{\bar{w}h}{h} \\
        & = \bar{w}. 
    \end{align*}
    Thus, $g(z)$ is holomorphic, and since $|z| < 1$ we know $g(z) \neq 0$. Thus, by the division rule of holomorphic functions, we have that $F = \frac{f}{g}$ is holomorphic. 
\end{proof}

\subsubsection*{ii)}
\begin{proof}
    We can quickly show that \[F(0) = \frac{w - 0}{1 - \bar{w} \cdot 0} = \frac{w}{1} = w \] 
    and, since $|w| < 1$ \[
        F(w) = \frac{w - w}{1 - \bar{w}w} = \frac{0}{1 - |w|^2} = 0 \].
\end{proof}

\subsubsection*{iii)}
\begin{proof}
    If $ z= 1$, then 
    \[
        |F(z)| = \left|\frac{w - z}{1 - \bar{w}z} \right| = 1
    \]
    by the first part of the problem. 
\end{proof}

\subsubsection*{iv)}
\begin{proof}
    We will prove that $F$ is bijective by proving that $F$ is its inverse. Thus, we will show $F\circ F(z) = z$ for $z \in \mathbb{D}$:
    \begin{align*}
        F \circ F(z) &= \frac{w - \frac{w - z}{1 - \bar{w}z}}{1 - \bar{w}\left(\frac{w - z}{1 - \bar{w}z} \right)} \\
        &= \frac{\frac{w - |w|^2z - w + z}{1 - \bar{w}z}}{1 - \frac{|w|^2 - \bar{w}z}{1 - \bar{w}z}} \\
        &= \frac{\frac{w - |w|^2z - w + z}{1 - \bar{w}z}}{\frac{1 - \bar{w}z - |w|^2 + \bar{w}z}{1 - \bar{w}z}}\\
        &= \frac{z - |w|^2z}{1 - |w|^2} = z \left(\frac{1 - |w|^2}{1 - |w|^2} \right)  \\
        &= z.
    \end{align*}
\end{proof}


\section*{9.}
\begin{proof}
    Let $z = x + iy = x(r, \theta) + iy(r, \theta) = re^{i \theta} = \cos \theta + i \sin \theta$. Let $f(z) = u(z(r, \theta), y(r, \theta)) + iv(x(r, \theta), y(r, \theta))$. We find that, 
    \begin{align*}
       \frac{\partial u}{\partial r} &\numeq{a} \frac{\partial u}{\partial x}\frac{\partial x}{\partial r} + \frac{\partial u}{\partial y}\frac{\partial y}{\partial r}  \\
   &= \frac{\partial u}{\partial x}\cos \theta + \frac{\partial u}{\partial y}\sin \theta \\
   &\numeq{b} \frac{\partial v}{\partial y}\cos \theta - \frac{\partial v}{\partial x}\sin \theta \\
   &= \frac{1}{r} \left(\frac{\partial v}{\partial y}r \cos \theta - \frac{\partial v}{\partial x}r \sin \theta \right) \\
   &= \frac{1}{r} \left(\frac{\partial v}{\partial y}\frac{\partial y}{ \partial \theta} + \frac{\partial v}{\partial x}\frac{\partial x}{\partial \theta}\right) \\
   &= \frac{1}{r} \frac{\partial v}{\partial \theta}. 
   \end{align*}
The steps $(a)-(b)$ are justified: 
\begin{enumerate}[label=(\alph*),leftmargin=2\parindent]
    \item chain rule of partial derivates
    \item Cauchy Riemann equations.
\end{enumerate}
To prove the other Cauchy Riemann equation, we use a similar process:
\begin{align*}
   \frac{1}{r}\frac{\partial u}{\partial \theta} &\numeq{a} \frac{1}{r} \left(\frac{\partial u}{\partial x} \frac{\partial x}{\partial \theta}+ \frac{\partial u}
   {\partial y}\frac{\partial u}{\partial y}\frac{\partial y}{\partial \theta}\right) \\
   &= \frac{1}{r} \left(-\frac{\partial u}{\partial x} r \sin \theta + \frac{\partial u}{\partial y} r \cos \theta \right) \\
   &= - \frac{\partial u}{\partial x} \sin \theta + \frac{\partial u}{\partial y} \cos \theta \\
   &= -\frac{\partial u}{\partial x}\frac{\partial y}{\partial r} + \frac{\partial u}{\partial y} \frac{x}{\partial r} \\
    &\numeq{b} = -\frac{\partial v}{\partial y} \frac{\partial y}{\partial r} - \frac{\partial v}{\partial x}\frac{\partial x}{\partial r} \\
    &= -\frac{\partial v}{\partial r}.
\end{align*}
The steps $(a)-(b)$ are justified: 
\begin{enumerate}[label=(\alph*),leftmargin=2\parindent]
    \item chain rule of partial derivates
    \item Cauchy Riemann equations.
\end{enumerate}
To check that $\log(z) = \log(r) + i \theta$ is holomorphic in $r > 0$ and $-\pi < \theta < \pi$ we can use our newly developed Cauchy Riemann equations. Take $u = \log(r)$ and $v = \theta$. We have that 
\[\frac{1}{r} \frac{\partial u}{\partial \theta} = 0 = -\frac{\partial v}{\partial \theta}\]
\[
    \frac{\partial y}{\partial r} = \frac{1}{r} = \frac{1}{r}\cdot 1 = \frac{1}{r} \frac{\partial v}{\partial \theta}.\]
    Thus, the Cauchy Riemann equations are satisfied, and because $u, v$ are clearly continuously differentiable for $r > 0$ and $-\pi < \theta < \pi$ we can apply the theorem stating that if a function satisfies the Cauchy Riemann equations and has $u, v$ continuously differentiable on a set, then it is holomorphic on the set to conclude that $\log(z)$ is holomorphic on $r > 0$ and $-\pi < \theta < \pi$. 
\end{proof}

\section*{13.}
\subsection*{a.}
\begin{proof}
    If $\Re(f) = c$ a constant, then that implies $u(x, y) = c$ and thus, using the Cauchy Riemann equations, \[
        \frac{\partial u}{\partial x} = 0 = \frac{\partial v}{\partial y} \text{ and } \frac{\partial u}{ \partial y} = 0 = -\frac{\partial v}{ \partial x}.\]
        Thus, $v(x, y) = c'$ a constant, and thus $f$ is a constant. 

\end{proof}
\subsection*{b.}
\begin{proof}
    In this case, we have that $v(x, y) = c$ a constant, and this assertion can be proved in the same way as part $(a)$ to show that $u(x, y) = c$ and hence $f$ is constant. 

\end{proof}

\subsection*{c.}
\begin{proof}
    If $|f| = \sqrt{u(x, y)^2 + v(x, y)^2} = c$ a constant, then $|f|^2 = u(x,y)^2 + v(x,y)^2 = c^2$ by properties of complex numbers. We can take the partial derivative of $|f|^2$ with respect to $x$ and $y$ and use the chain rule for partial derivatives to get two equations:
    \begin{align*}
        0 &= 2u(x, y)\frac{\partial u}{\partial x} + 2v(x, y)\frac{\partial v}{\partial x}\\
        0 &= 2u(x, y)\frac{\partial u}{\partial y} + 2v(x, y)\frac{\partial v}{\partial y}.
    \end{align*}
Multiplying by $2$ on each side, we can simplify our equations to 
\begin{align*}
    0 &= u(x, y)\frac{\partial u}{\partial x} + v(x, y)\frac{\partial v}{\partial x}\\
    0 &= u(x, y)\frac{\partial u}{\partial y} + v(x, y)\frac{\partial v}{\partial y}.
\end{align*}
Now, we can multiply the first equation by $u(x, y)$ and the second equation by $v(x, y)$, which will help us simplify it shortly:
\begin{align*}
    0 &= u(x, y)^2\frac{\partial u}{\partial x} + u(x,y)v(x, y)\frac{\partial v}{\partial x}\\
    0 &= u(x, y)v(x, y)\frac{\partial u}{\partial y} + v(x, y)^2\frac{\partial v}{\partial y}.
\end{align*}
Now, we can use the Cauchy-Riemann equations to simplify further
\begin{align*}
    0 &= u(x, y)\frac{\partial u}{\partial x} + v(x, y)\frac{\partial v}{\partial x}\\
    0 &= -u(x, y)\frac{\partial v}{\partial x} + v(x, y)\frac{\partial v}{\partial y}.
\end{align*}
Combining the two equations, we obtain
\begin{align*}
    0 u(x, y)^2 \frac{\partial u}{\partial x} + v(x, y)^2 \frac{\partial v}{\partial y}.
\end{align*}
Once again, using the Cauchy-Riemann equation, we get that 
\begin{align*}
    0 = u(x, y)^2\frac{\partial u}{\partial x} + v(x, y)^2 \frac{\partial u}{ \partial x}
\end{align*}
this implies that 
\begin{align*}
    0 = \frac{\partial u}{\partial x}(u(x, y)^2 + v(x,y)^2).
\end{align*}
Either $u(x, y)^2$ and $v(x, y)^2 = 0$ in which case $f = 0$ or $\frac{\partial u}{\partial x} = 0$. Similarly, it can be shown that 
$\frac{\partial u}{\partial y}$, $frac{\partial v}{\partial x}$, and $\frac{\partial v}{\partial y}$ must be $0$ or the function must be $0$, and hence $f$ is a constant. 




\end{proof}


\section*{14.}
\begin{proof}
    Assume that $0 \leq M \leq N$ as otherwise, the equation will always equal $0$. We will manipulate the series in several ways to show the equality:
    \begin{align*}
        \sum\limits_{n = M}^N a_nb_n &= \sum\limits_{n = M}^Na_n \left(\sum\limits_{i = 1}^n b_i - \sum\limits_{i = 1}^{n - 1} b_i \right) \\
        &= \sum\limits_{n = M}^N a_n (B_n - B_{n - 1}) \\
        &= \sum\limits_{n = M}^N a_n B_n - \sum\limits_{n = M}^N a_n B_{n - 1} \\
        &= a_NB_N + \sum\limits_{n = M}^{N - 1}a_n B_n - a_MB_{M - 1} - \sum\limits_{n = M + 1}^Na_nB_{n - 1} \\
        &\numeq{a} a_NB_N - a_MB_{M - 1} + \sum\limits_{n = M}^{N - 1}a_n B_n - \sum\limits_{n = M}^{N - 1}a_{n + 1}B_n \\
        &= a_NB_N - a_MB_{M - 1} + \sum\limits_{n = M}^{N - 1}(a_n - a_{n + 1}) B_n \\
        &= a_NB_N - a_MB_{M - 1} - \sum\limits_{n = M}^{N - 1}(a_{n + 1} - a_n)B_n. 
    \end{align*}


    Step (a) is justified as with the convention that $B_0 = 0$, we can shift the indices to get that $\sum\limits_{n = M + 1}^Na_nB_{n - 1} = \sum\limits_{n = M}^{N - 1}a_{n + 1}B_n$. 
\end{proof}

\section*{19.}
\subsection*{a.}
\begin{proof}
Let $a_n = nz^n$. For a series to converge, from real analysis, we know that we need $\lim\limits_{n \to \infty} |a_n| \to 0$. But, \[\lim\limits_{n \to \infty} |a_n| = \lim\limits_{n \to \infty} |nz^n| = \lim\limits_{n \to infty} n |z|^n = \lim\limits_{n \to \infty} n = \infty. \] 
Thus, the series does not converge for $|z| = 1$. 
\end{proof}

\subsection*{b.}
\begin{proof}
    We will use the fact that if a series converges absolutely, it converges. Thus, $\sum\left|\frac{1}{n^2} \right||z|^n = \sum\frac{1}{n^2}$ which we know converges. 
    Thus, our series converges for $|z| = 1$. 
\end{proof}

\subsection*{c.}
\begin{proof}
Let $|z| = 1$. If $z = 1$, then $\sum \frac{z^n}{n} = \sum\frac{1}{n}$ which we know diverges. So, suppose $z \neq 1$. Let $S_N = \sum\limits_{n = 1}^N a_nb_n$, where $a_n = \frac{1}{n}$ and $b_n = z^n$. Also, let $B_N = \sum\limits_{n = 1}^N b_n$. By the summation by parts proven earlier we have that \[S_N = a_{N}B_{N} - a_1B_0 - \sum\limits_{n = 1}^{N - 1}(a_{n + 1}- a_n)B_n.
\]
Since, $B_0 = 0$ and $- \sum\limits_{n = 1}^{N - 1}(a_{n + 1}- a_n)B_n = \sum\limits_{n = 1}^{N - 1}(a_n - a_{n + 1})B_n$, we get 
\[
    S_N = a_{N}B_{N} + \sum\limits_{n = 1}^{N - 1}(a_n - a_{n + 1})B_n. 
\]
We need to show that $B_N$ is bounded for every $N$. But, this is true, as $B_N = \sum\limits_{n = 1}^{N}e^{in\theta} = \frac{1 - e^{i \theta(N + 1)}}{1 - e^{i \theta}}$ as shown in the textbook, and this is bounded for all $N$. 
Because $|B_N|$ is bounded for every $N$, as $|z| = 1$ let the bound for an arbitrary $B_n$ be denoted by $M$. Therefore, as $a_n = \frac{1}{n} \to 0$ we know that $a_NB_N = \frac{1}{n}\sum\limits_{n = 1}^Nz^n \to 0$ becuase $B_N$ is bounded for all $N$ by $M$. Our sequence is monotone decreasing, so 
\[
   \sum\limits_{n = 1}^{N - 1}\left|(a_n - a_{n + 1})B_n\right|  \leq \sum\limits_{n = 1}^{N - 1} \left|(a_n - a_{n + 1}) \right|M \numeq{a} \sum\limits_{n = 1}^{N - 1} (a_n - a_{n + 1})M
    \]
    where step (a) is true because the sequence is monotone decreasing. Thus, because 
    \[
        \sum\limits_{n = 1}^{N - 1} (a_n - a_{n + 1})M
        \]
    is a telescoping sum
    \[
        \lim\limits_{n \to \infty} \sum\limits_{n = 1}^{N - 1} (a_n - a_{n + 1})M = a_1M.
        \]

    We can conclude that 
    \[
        \lim\limits_{n \to \infty} S_N = \lim\limits_{n \to \infty} \left( a_{N}B_{N} + \sum\limits_{n = 1}^{N - 1}(a_n - a_{n + 1})B_n  \right)= 0 + \lim\limits_{n \to \infty} \sum\limits_{n = 1}^{N - 1}(a_n - a_{n + 1})B_n \leq \lim\limits_{n \to \infty} \sum\limits_{n = 1}^{N - 1}(a_n - a_{n + 1})M = a_1M.  
        \]
\end{proof}


\section*{23.}
\begin{proof}
    Let $p(x)$ be a polynomial. Our claim is that for $x > 0$, where $f(x) = e^{-\frac{1}{x^2}}$ that $f^{(n)}(x)$ is of the form. 
    We will prove this assertion by induction. For $f^{(1)}(x)$ we have 
    \[f^{(1)}(x) = \frac{2}{x^3}e^{-\frac{1}{x^2}}\] 
    which is of the form $p(x)e^{-\frac{1}{x^2}}$ and is defined since $x > 0$. Now, assume the claim holds for all $n$, and we will prove the $n + 1$ case. For $f^{(n + 1)}(x)$ we have that by our hypothesis we know that $f^{(n)} = p(x)e^{-\frac{1}{x^2}}$, so we then have that $f^{(n + 1)}(x) = p(x) \frac{2}{x^3}e^{-\frac{1}{x^2}}$. So, the function is indefinitely differentiable
    for $x > 0$, and if $x\leq 0$, then $f(x) = 0 $, so clearly, every derivative will be just $0$. Thus, $f$ is indefinitely differentiable for any $x \in \mathbb{R}$. We know that $f^{(n)}(x) = 0$ for any $x \leq 0$, so it follows that $f^{(n)}(0) = 0$. 
    Therefore, looking at the power series expansion for $e^{-\frac{1}{x^2}}$ we see 
    \[
    e^{-\frac{1}{x^2}} = \sum\limits_{n = 0}^{\infty}\frac{\left(-\frac{1}{x^2} \right)^n}{n!} = 1 - \frac{1}{x^2} + \frac{1}{2!}\frac{1}{x^4} - \frac{1}{3!}\frac{1}{x^6} + \cdots    
    \] 
    which is undefined for the value $x = 0$, and hence, we should expect no converging power series expansion near the origin to exist. 
\end{proof}
                            
\end{document}